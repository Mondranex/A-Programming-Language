
\chapter{The Language}

\section{Introduction}

\par Applied mathematics is concerned with the design and analysis of algorithms or programs. The systematic treatment of complex algorithms requires a suitable programming language for their description, and such a programming language should be concise, precise, consistent over a wide area of application, mnemonic, and economical of symbols; it should exhibit clearly the constraints on the sequence in which operations are performed; and it should permit the description of a process to be independent of the particular representation chosen for the data.

\par Existing languages prove unsuitable for a variety of reasons. Computer coding specifies sequence constraints adequately and is also comprehensive, since the logical functions provided by the branch instructions can, in principle, be employed to synthesize any finite algorithm. However, the set of basic operations provided is not, in general, directly suited to the execution of commonly needed processes, and the numeric symbols used for variables have little mnemonic value. Moreover, the description provided by computer coding depends directly on the particular representation chosen for the data, and it therefore cannot serve as a description of the algorithm per se.

\par Ordinary English lacks both precision and conciseness. The widely used Goldstine-von Neumann (1947) flowcharting provides the conciseness necessary to an over-all view of the processes, only at the cost of suppressing essential detail. The so-called pseudo-English used as a basis for certain automatic programming systems suffers from the same defect. Moreover, the potential mnemonic advantage in substituting familiar English words and phrases for less familiar but more compact mathematical symbols fails to materialize because of the obvious but unwonted precision required in their use.

\par Most of the concepts and operations needed in a programming language have already been defined and developed in one or another branch of mathematics. Therefore, much use can and will be made of existing notations. However, since most notations are specialized to a narrow field of discourse, a consistent unification must be provided. For example, separate and conflicting notations have been developed for the treatment of sets, logical variables, vectors, matrices, and trees, all of which may, in the broad universe of discourse of data processing, occur in a single algorithm. 

\section{Programs}

\par A \textit{program statement} is the specification of some quantity or quantities in terms of some finite operation upon specified operands. Specification is symbolized by an arrow directed toward the specified quantity. thus "y is specified by sin x" is a statement denoted by

$$
      y\ ←\ \sin \ x.
$$

\par A set of statements together with a specified order of execution constitutes a \textit{program}. The program is finite if the number of executions is \textit{finite}. The \textit{results} of the program are some subset of the quantities specified by the program. The \textit{sequence} or order of execution will be defined by the order of listing and otherwise by arrows connecting any statement to its successor. A cyclic sequence of statements is called a \textit{loop}.

\par (TODO: FIGURES 1.1 AND 1.2)

\par Thus Program 1.1 is a program of two statements defining the result $v$ as the (approximate) area of a circle of radius $x$, whereas Program 1.2 is an infinite program in which the quantity $z$ is specified as $(2y)^n$ on the $n$-th execution of the two statement loop. Statements will be numbered on the left for reference.

\par A number of similar programs may be subsumed under a single more general program as follows. At certain \textit{branch points} in the program a finite number of alternative statements are specified as possible successors. One of these successors is chosen according to criteria determined in the statement or statements preceding the branch point. These criteria are usually stated as a \textit{comparison} or test of a specified relation between a specified pair of quantities. A branch is denoted by a set of arrows leading to each of the alternative successors, with each arrow labeled by the comparison condition under which the corresponding successor is chosen. The quantities compared are separated by a colon in the statement at the branch point, and a labeled branch is followed if and only if the relation indicated by the label holds when substituted for the colon. The conditions on the branches of a properly defined program must be disjoint and exhaustive.

\par Program 1.3 illustrates the use of a branch point. Statement \verb|α|$5$ is a comparison which determines the branch to statements \verb|β|$1$, \verb|δ|$1$, or \verb|γ|$1$, according as $z > n$, $z = n$, or $z < n$. The program represents a crude by effective process for determining $x = n^\frac{2}{3}$ for any positive cube n.

\par (TODO: FIGURE 1.3)

\par Program 1.4 shows the preceding program reorganized into a compact linear array and introduces two further conventions on the labeling of branch points. The listed successor of a branch statement is selected if none of the labeled conditions is met. Thus statement 6 follows statement 5 if neither of the arrows (to exit or to statement 8) are followed, i.e. if $z < n$. Moreover, any unlabeled arrow is always followed; e.g., statement 7 is invariably followed by statement 3, never by statement 8.

\par A program begins at a point indicated by an \textit{entry arrow} (step 1) and ends at a point indicated by an \textit{exit arrow} (step 5). There are two useful consequences of confining a program to the form of a linear array: the statements may be referred to by a unique serial index (statement number), and unnecessarily complex organization of the program manifests itself in crossing branch lines. The importance of the latter characteristic in developing clear and comprehensible programs is not sufficiently appreciated.

\par (TODO: FIGURES 1.4 AND 1.5)

\par A process which is repeated a number of times is said to be iterated, and a process (such as in Program 1.4) which includes one or more iterated subprocesses is said to be iterative. Program 1.5 shows an iterative process for the matrix multiplication

$$
      C\ ←\ AB
$$

\noindent defined in the usual way as

\begin{equation*}
  \begin{split}
    C^i_j = \sum^{v(\textbf{A})}_{k=1} A^k_i \times B^k_j
  \end{split}
\quad\quad
  \begin{split}
    i &= 1,2, ..., \mu(\textbf{A}),\\
    j &= 1,2, ..., v(\textbf{B}),
  \end{split}
\end{equation*}

\noindent where the dimensions of an $m \times n$ matrix $X$ (of $m$ rows and $n$ columns) is denoted by $\mu(X) \times v(X)$.

\par \textbf{Program 1.5.} Steps 1-3 initialize the indices, and the loop 5-7 continues to add successive products to the partial sum until k reaches zero. When this occurs, the process continues through step 8 to decrement $j$ and to repeat the entire summation for the new value of $j$, providing that it is not zero. If $j$ is zero, the branch to step 10 decrements $i$ and the entire process over $j$ and $k$ is repeated from $j = v(\textbf{B})$, providing that $i$ is not zero. If $i$ is zero, the process is complete, as indicated by the exit arrow.

\par In all examples used in this chapter, emphasis will be placed on clarity of description of the process, and considerations of efficient execution by a computer or class of computers will be subordinated. These considerations can often be introduced later by relatively routine modifications of the program. For example, since the execution of a computer operation involving an indexed variable is normally more costly than the corresponding operation upon a nonindexed variable, the substitution of a variable s for the variable $C^i_j $ specified by statement 5 of Program 1.5 would accelerate the execution of the loop. The variable s would be initialized to zero before each entry to the loop and would be used to specify $C^i_j$ at each termination.

\par The practice of first setting an index to its maximum value and then decrementing it (e.g., the index $k$ in Program 1.5) permits the termination comparison to be made with zero. Since zero often occurs in comparisons, it is convenient to omit it. Thus, if a variable stands alone at a branch point, comparison with zero is implied. Moreover, since a comparison on an index frequently occurs immediately after it is modified, a branch at the point of modification will denote branching upon comparison of the indicated index with zero, the comparison occurring \textit{after} modification. Designing programs to execute decisions immediately after modification of the controlling variable results in efficient execution as well as notational elegance, since the variable must be present in a central register for both operations.

\par Since the sequence of execution of statements is indicated by connecting arrows as well as by the order of listing, the latter can be chosen arbitrarily. This is illustrated by the functionally identical Programs 1.3 and 1.4. Certain principles of ordering may yield advantages such as clarity or simplicity of the pattern of connections. Even though the advantages of a particular organizing principle are not particularly marked, the uniformity resulting from its consistent application will itself be a boon. The scheme here adopted is called the \textit{method of leading decisions}: the decision on each parameter is placed as early in the program as practicable, normally just before the operations indexed by the parameter. This arrangement groups at the head of each iterative segment the initialization, modification, and the termination test of the controlling parameter. Moreover, it tends to avoid program flaws occasioned by unusual values of the argument.

\par (TODO: FIGURE 1.6)

\par For example, Program 1.6 (which is a reorganization of Program 1.5) behaves properly for matrices of dimension zero, whereas Program 1.5 treats every matrix as if it were of dimension one or greater.

\par Although the labeled arrow representation of program branches provides a complete and graphic description, it is deficient in the following respects: (1) a routine translation to another language (such as computer code) would require the tracing of arrows, and (2) it does not permit programmed modification of the branches.

\par The following alternative form of a branch statement will therefore be used as well:

$$
x\ :\ y,\ r\ →\ s.
$$

\par This denotes a branch to statement number $s_i$ of the program if the relation $xr_i y$ holds. The parameters $r$ and $s$ may themselves be defined and redefined in other parts of the program. The null element $\circ$ will be used to denote the relation which complements the remaining relations $r_i$; in particular, $(\circ)\ →\ (s)$, or simply $→\ s$, will denote an unconditional branch to statement $s$. Program 1.7 shows the use of these conventions in a reformulation of Program 1.6. More generally, two or more otherwise independent programs may interact through a statement in one program specifying a branch in a second. The statement number occurring in the branch must then be augmented by the name of the program in which the branch is effected. Thus the statement $(\circ)\ →$ Program 2.24 executed in Program 1 causes a branch to step 24 to occur in Program 2.

\par (TODO: FIGURE 1.7)

\par One statement in a program can be modified by another statement which changes certain of its parameters, usually indices. More general changes in statements can be effected by considering the program itself as a vector $p$ whose components are the individual, serially numbered statements. All the operations to be defined on general vectors can then be applied to the statements themselves. For example, the $j$-th statement can be respecified by the $i$-th through the occurrence of the statement $p_j\ ←\ p_i$.

\par The interchange of two quantities $y$ and $x$ (that is, $x$ specifies $y$ and the original value of $y$ specifies $x$) will be denoted by the statement $y\ \leftrightarrow\ x$.

\section{Structure of the language}

\subsection*{Conventions}

\par The Summary of Notation at the end of the book summarizes the notation developed in this chapter. Although intended primarily for reference, it supplements the text in several ways. It frequently provides a more concise alternative definition of an operation discussed in the text, and it also contains important but easily grasped extensions not treated explicitly in the text. By grouping the operations into related classes it displays their family relationships.

\par A concise programming language must incorporate families of operations whose members are related in a systematic manner. Each family will be denoted by a specific operation symbol, and the particular member of the family will be designated by an associated \textit{controlling parameter} (scalar, vector, matrix, or tree) which immediately precedes the main operation symbol. The operand is placed immediately after the main operation symbol. For example, the operation $k\ ↑\ x$ (left rotation of $x$ by $k$ places) may be viewed as the $k$-th member of the set of rotation operators denoted by the symbol $↑$.

\par Operations involving a single operand and no controlling parameter (such as $\lfloor x \rfloor$, or $\lceil x \rceil$) will be denoted by a pair of operation symbols which enclose the operand. Operations involving two operands and a controlling parameter (such as the mask operation $/a,\ u,\ b/$) will be denoted by a pair of operation symbols enclosing the entire set of variables, and the controlling parameter will appear between the two operands. In these cases the operation symbols themselves serve as grouping symbols.
