
\chapter{Summary of Notation}

\section{Conventions}

\subsection{Basic conventions}
\begin{enumerate}[label= (\alph*)]
	\item 1-origin indexing assumed in this summary.

	\item Controlling variables appear to the left, e.g., \( \vect{u} / \vect{x} \), \( \vect{b} ⊥ \vect{y} \), \( k ↑ \vect{x} \), and \( \vect{u} ⌈ \vect{x} \).

	\item Dimension \( n \) may be elided (if determined by compatibility) from \( \bfepsilon(n) \), \( \bfepsilon^k(n) \), \( \bfalpha^k(n) \), \( \bfomega^k(n) \), and \( \bfiota^j(n) \).

	\item The parameter \( j \) may be elided from operators \( |_j \), \( θ_j \), \( ∫_j \), and \( ι_j \), and from the vector \( \bfiota^j \) if \( j \) is the index origin in use.

	\item The parameter \( k \) may be elided from \( k↑\vect{x} \) if \( k=1 \).
\end{enumerate}

\subsection{Branching conventions}
\begin{enumerate}[label= (\alph*)]
	\item \( \left| x : y \right| \overset{\symscr{R}}{\rightarrow} \)\newline
	The statement to which the arrow leads is executed next if \( \left( x \symscr{R} y \right) = 1 \); otherwise the listed successor is executed next. An unlabeled arrow is always followed.
	\item \( x:y, \vect{r} → \vect{s} \)\newline
	The statement numbered \( \vect{s}_i \) is executed next if \( \left( x \vect{r}_i y \right) = 1 \). The null symbol \( ∘ \) occurring as a component of \( \vect{r} \) denotes the relation which complements the disjunction of the remaining relations in \( \vect{r} \).
	\item \( → \symup{Program}\ a,b \)\newline
	Program \( a \) branches to its statement \( b \). The symbol \( a \) may be elided if the statement occurs in Program \( a \) itself.
\end{enumerate}

\subsection{Operand conventions used in summary}
\begin{tabularx}{\textwidth}{ l l l l l }
	% empty cell
		& Scalar
		& Vector
		& Matrix
		& Tree
		\\
	Logical
		& \( u,v,w \)
		& \( \vect{u,v,w} \)
		& \( \mat{U,V,W} \)
		& \( \tree{U,V,W} \)
		\\
	Intergral
		& \( h,i,j,k \)
		& \( \vect{h,i,j,k} \)
		& \( \mat{H,I,J,K} \)
		& \( \tree{H,I,J,K} \)
		\\
	Numerical
		& \( x,y,z \)
		& \( \vect{x,y,z} \)
		& \( \mat{X,Y,Z} \)
		& \( \tree{X,Y,Z} \)
		\\
	Aribitrary
		& \( a,b,c \)
		& \( \vect{a,b,c} \)
		& \( \mat{A,B,C} \)
		& \( \tree{A,B,C} \)
		\\
\end{tabularx}

\section{Structural Parameters, Null}
\begin{tabularx}{\textwidth}{ l c X }
	Dimension
		& \( \nu(\vect{a}) \)
		& Number of components in vector \( \vect{a} \)
		\\
	Row dimension
		& \( \nu(\mat{A}) \)
		& Number of components in each row vector \( \mat{A}^i \)
		\\
	Column dimension
		& \( \mu(\mat{A}) \)
		& Number of components in each column vector \( \mat{A}_j \)
		\\
	Height
		& \( \nu(\tree{A}) \)
		& Length of longest path in \( \tree{A} \)
		\\
	Moment
		& \( \mu(\tree{A}) \)
		& Number of nodes in \( \tree{A} \)
		\\
	Dispersion vector
		& \( \bfnu(\tree{A}) \)
		& \( \bfnu_1(\tree{A}) = \) number of roots of \( \tree{A} \); \( \bfnu_j(\tree{A}) = \) maximum degree of nodes on level \( j − 1 \); \( \nu(\bfnu(\tree{A})) = \nu(\tree{A}) \)
		\\
	Moment vector
		& \( \bfmu(\tree{A}) \)
		& \( \bfmu_j(\tree{A}) = \) number of nodes on level \( j \) of \( \tree{A} \); \( \nu(\bfmu(\tree{A})) = \nu(\tree{A}) \)
		\\
	Degree of node \( \vect{i} \)
		& \( \delta(\vect{i}, \tree{A}) \)
		& Degree of node \( \vect{i} \) of tree \( \tree{A} \)
		\\
	Degree
		& \( \delta(\tree{A}) \)
		& \( \delta(\tree{A}) = \underset{\vect{i}}{\max}\ \delta(\vect{i},\tree{A}) \)
		\\
	Leaf count
		& \( \lambda(\tree{A}) \)
		& \( \lambda(\tree{A}) \) is the number of leaves in \( \tree{A} \)
		\\
	Row dimension of file
		& \( \nu(\mathbf{Φ}) \)
		& Number of files in each row of a file array
		\\
	Column dimension of file
		& \( \mu(\mathbf{Φ}) \)
		& Number of files in each column of a file array
		\\
	Null character
		& \( ∘ \)
		& Null character of set (e.g., space in the alphabet) or null reduction operator
		\\
\end{tabularx}

\section{Relations}


\section{Elementary Operations}


\section{Vector Operations}


\section{Row Generalizations of Vector Operations}


\section{Column Generalizations of Vector Operations}


\section{Special Matrices}


\section{Transposition}


\section{Set Operations}


\section{Generalized Matrix Product}


\section{Files}


\section{Trees}

