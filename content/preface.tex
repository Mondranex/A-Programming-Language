
\chapter*{PREFACE}

\par Applied mathematics is largely concerned with the design and analysis of explicit procedures for calculating the exact or approximate values of various functions. Such explicit procedures are called algorithms or \textit{programs}. Because an effective notation for the description of programs exhibits considerable syntactic structure, it is called a \textit{programming language}.

\par Much of applied mathematics, particularly the more recent computer-related areas which cut across the older disciplines, suffers from the lack of an adequate programming language. It is the central thesis of this book that the descriptive and analytic power of an adequate programming language amply repays the considerable effort required for its mastery. This thesis is developed by first presenting the entire language and then applying it in later chapters to several major topics.

\par The areas of application are chosen primarily for their intrinsic interest and lack of previous treatment, but they are also designed to illustrate the universality and other facets of the language. For example, the microprogramming of Chapter 2 illustrates the divisibility of the language, i.e., the ability to treat a restricted area using only a small portion of the complete language. Chapter 6 (Sorting) shows its capacity to compass a relatively complex and detailed topic in a short space. Chapter 7 (The Logical Calculus) emphasizes the formal manipulability of the language and its utility in theoretical work.

\par The material was developed largely in a graduate course given for several years at Harvard and in a later course presented repeatedly at the IBM Systems Research Institute in New York. It should prove suitable for a two-semester course at the senior or graduate level. Although for certain audiences an initial presentation of the entire language may be appropriate, I have found it helpful to motivate the development by presenting the minimum notation required for a given topic, proceeding to its treatment (e.g., microprogramming), and then returning to further notation. The 130-odd problems not only provide the necessary finger exercises but also develop results of general interest.

\par Chapter 1 or some part of it is prerequisite to each of the remaining ``applications'' chapters, but the applications chapters are virtually independent of one another. A complete appreciation of search techniques (Chapter 4) does, however, require a knowledge of methods of representation (Chapter 3). The cross references which do occur in the applications chapters are either nonessential or are specific to a given figure, table, or program. The entire language presented in Chapter 1 is summarized for reference at the end of the book.

\par In any work spanning several years it is impossible to acknowledge adequately the many contributions made by others. Two major acknowledgments are in order: the first to Professor Howard Aiken, Director Emeritus of the Harvard Computation Laboratory, and the second to Dr. F.P. Brooks, Jr. now of IBM.

\par It was Professor Aiken who first guided me into this work and who provided support and encouragement in the early years when it mattered. The unusually large contribution by Dr. Brooks arose as follows. Several chapters of the present work were originally prepared for inclusion in a joint work which eventually passed the bounds of a single book and evolved into our joint \textit{Automatic Data Processing} and the present volume. Before the split, several drafts of these chapters had received careful review at the hands of Dr. Brooks, reviews which contributed many valuable ideas on organization, presentation, and direction of investigation, as well as numerous specific suggestions.

\par The contributions of the 200-odd students who suffered through the development of the material must perforce be acknowledged collectively, as must the contributions of many of my colleagues at the Harvard Computation Laboratory. To Professor G.A. Salton and Dr. W.L. Eastman, I am indebted for careful reading of drafts of various sections and for comments arising from their use of some of the material in courses. Dr. Eastman, in particular, exorcised many subtle errors from the sorting programs of Chapter 6. To Professor A.G. Oettinger and his students I am indebted for many helpful discussions arising out of his early use of the notation. My debt to Professor R.L. Ashenhurst, now of the University of Chicago, is apparent from the references to his early (and unfortunately unpublished) work in sorting.

\par Of my colleagues at the IBM Research Center, Messrs. L.R. Johnson and A.D. Falkoff, and Dr. H. Hellerman have, through their own use of the notation, contributed many helpful suggestions. I am particularly indebted to L.R. Johnson for many fruitful discussions on the applications of trees, and for his unfailing support.

\par On the technical side, I have enjoyed the assistance of unusually competent typists and draughtsmen, chief among them being Mrs. Arthur Aulenback, Mrs. Philip J. Seaward, Jr., Mrs. Paul Bushek, Miss J.L. Hegeman, and Messrs. William Minty and Robert Burns. Miss Jacquelin Sanborn provided much early and continuing guidance in matters of style, format, and typography. I am indebted to my wife for assistance in preparing the final draft.

\begin{flushright}
\textsc{Kenneth E. Iverson}
\end{flushright}

\noindent \textit{May, 1962 \\
Mount Kisco, New York}
